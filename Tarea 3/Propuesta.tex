\documentclass{article}
\usepackage[utf8]{inputenc}
\usepackage{graphicx}


\begin{document}

\section*{Introducción}
Las PWA son una herramienta que día con día va creciendo. A través del tiempo se ha demostrado que cada vez más, las herramientas se mueven a entornos cada vez más pequeños, más sencillos y menos robustos. Tras todo eso hacer los sistemas multiplataformas, que generen menos gastos, actualizaciones constantes y la persistencia de los datos en entornos de seguridad es muy importante para reaccionar antes cualquier pérdida de conexión y flexibilidad para el sistema.

De formas que crear una PWA para nuestro proyecto puede tener bastantes beneficios para todos los empleados que estén usando el sistema, así como reducir grandes cantidades de costos. Además de normalizar un solo entorno donde las personas que programen sean más proactivas y tengan que preocuparse menos por crear un servicio desde cero.

\clearpage

\section*{¿Por qué elegir una PWA para nuestro proyecto?}
Elegir una PWA para la realización del proyecto es un acierto debido al objetivo y meta que nosotros tenemos hacia el funcionamiento del producto, la flexibilidad para los desarrolladores, la comodidad de los usuarios y el bajo costo al igual que tiempo para implementarlo en nuestro proyecto.

Facilitar el uso del software para los guardias de seguridad a la hora de atender sus tickets en el sistema de seguridad garantizará que el sistema esté funcionando correctamente y que no hay nada inusual en el sistema integrado. De igual manera, mejorará la flexibilidad al momento de movilizarse entre cada uno de los departamentos donde esté el sistema, haciendo que pueda atender cualquier cosa en esos tiempos de recorrido.

\subsection*{Objetivo del proyecto}
Controlar las entradas y salidas a ubicaciones específicas de diferentes departamentos en una empresa, resguardando las herramientas, consumibles, documentación que deseen proteger.

%\begin{figure}[h]
%    \centering
%    \includegraphics[width=0.7\textwidth]{ruta/de/la/imagen}
%    \caption{Descripción de la imagen.}
%    \label{fig:imagen}
%\end{figure}

\clearpage

\section*{Usabilidad al agregar nuevos dispositivos}
En el contexto de un entorno de seguridad, la eficacia y la simplicidad en la gestión de dispositivos son esenciales para garantizar un sistema robusto y confiable. La adopción de una Progressive Web App (PWA) para la incorporación de nuevos dispositivos, como sensores de movimiento y sensores de huellas, puede ser la clave para simplificar y mejorar significativamente el proceso de instalación.

Imaginemos a nuestro técnico encargado de la instalación, llevando consigo únicamente su teléfono celular. Con la PWA, el proceso se inicia directamente desde su dispositivo, sin la necesidad de descargar aplicaciones adicionales. Esto no solo ahorra tiempo, sino que también simplifica la gestión de dispositivos, permitiendo al técnico centrarse en la tarea en cuestión en lugar de lidiar con procesos de instalación complejos. La interfaz intuitiva de la PWA guía al técnico a través de cada paso de la instalación, facilitando la configuración de los nuevos dispositivos con unos pocos clics. La compatibilidad multiplataforma asegura que el técnico pueda utilizar su propio dispositivo, independientemente de la plataforma, eliminando las limitaciones y facilitando la adopción generalizada.

Además, la PWA ofrece beneficios prácticos como actualizaciones instantáneas y funcionamiento offline. Las actualizaciones son implementadas de manera automática, asegurando que el técnico siempre cuente con la versión más reciente de la herramienta de instalación. En situaciones donde la conectividad es limitada, la PWA sigue siendo completamente funcional, permitiendo al técnico completar la instalación sin interrupciones. La seguridad y la privacidad son de suma importancia en sistemas de seguridad, y la PWA no escatima en estos aspectos. Cumple con altos estándares de seguridad, proporcionando un entorno confiable para la incorporación de dispositivos sensibles.

\clearpage

\section*{Conexión local}
En el siguiente enunciado veremos la ventaja que tendríamos en una situación donde la conectividad con internet es mala y necesitamos seguir midiendo los datos de los sensores en este tiempo donde se pierde la conexión.

Nuestro técnico de seguridad se encuentra en un entorno donde la conectividad a Internet es intermitente o incluso inexistente. Aquí es donde la magia de las Progressive Web Apps (PWAs) entra en juego, ofreciendo ventajas significativas que pueden revolucionar la forma en que registramos y actualizamos sensores, especialmente en zonas sin conexión. Una de las principales ventajas de las PWAs es su capacidad para funcionar sin conexión. Esto significa que nuestro técnico puede registrar nuevos sensores de movimiento y sensores de huellas incluso en áreas remotas, sin depender de la disponibilidad de una red. La PWA proporcionará una experiencia de usuario fluida y permitirá al técnico realizar todas las funciones esenciales para la instalación sin obstáculos.

Ahora, aquí está la genialidad: cuando el técnico se desplaza a una zona con conexión a Internet, la PWA tiene la capacidad

\end{document}
